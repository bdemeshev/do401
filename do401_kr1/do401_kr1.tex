\documentclass[pdftex,12pt,a4paper]{article}

%\input{/home/boris/science/tex_general/title_bor_utf8}
\usepackage[utf8]{inputenc} % задание utf8 кодировки исходного tex файла
\usepackage[russian]{babel}
\usepackage{amsmath}
\usepackage[paper=a4paper,top=13.5mm, bottom=13.5mm,left=16.5mm,right=13.5mm,includefoot]{geometry}

\usepackage{enumitem}

\setlist[enumerate,1]{label=\arabic*., ref=\arabic*, partopsep=0pt plus 2pt, topsep=0pt plus 1.5pt,itemsep=0pt plus .5pt,parsep=0pt plus .5pt}
\setlist[itemize,1]{partopsep=0pt plus 2pt, topsep=0pt plus 1.5pt,itemsep=0pt plus .5pt,parsep=0pt plus .5pt}


\newcommand{\shpargalka}{Шпаргалка

\begin{enumerate}
\item Позор джунглям! $\sin'(x)=\cos(x)$ и $\cos'(x)=-\sin(x)$

\item Уравнение Эйлера-Пуассона
\[
F_y-\frac{d}{dt}F_{y'}+\frac{d^2}{dt^2}F_{y''}+\ldots+(-1)^n\frac{d^n}{dt^n}F_{y^{(n)}}=0
\]

\item Для автономной системы $v[y]=\int_a^b F(y,y')\, dt \to extr$ уравнение Эйлера приводится к виду $y'F_{y'}-F=const$

\item Условие трансверсальности 
\[
[F-y'F_{y'}]_{t=T}\cdot \Delta T + [F_{y'}]_{t=T}\cdot \Delta y_T=0
\]

\item Условие трансверсальности для задачи Больца
\[
\begin{cases}
\left. F_{y'}\right|_{t_0}=G_{y(t_0)} \\
\left. F_{y'}\right|_{t_1}=-G_{y(t_1)}
\end{cases}
\]

\item Решение линейных дифференциальных уравнений

\begin{enumerate}
\item Общее решение однородного уравнения $y^{(n)}+\ldots+a_1 y'+a_0 y=0$
\begin{enumerate}
\item Вещественный корень $\lambda$ кратности $k$: $e^{\lambda t}$, $te^{\lambda t}$, $t^2e^{\lambda t}$, \ldots, $t^{k-1}e^{\lambda t}$
\item Пара комплексных корней $\lambda=a\pm bi$  кратности $k$:

$e^{at}\cos(bt)$, $te^{at}\cos(bt)$, $t^2e^{at}\cos(bt)$, \ldots, $t^{k-1}e^{at}\cos(bt)$ \\
$e^{at}\sin(bt)$, $te^{at}\sin(bt)$, $t^2e^{at}\sin(bt)$, \ldots, $t^{k-1}e^{at}\sin(bt)$
\end{enumerate}
\item Частное решение неоднородного уравнения $y^{(n)}+\ldots+a_1 y'+a_0 y=f(t)$
\begin{enumerate}
\item Если $f(t)=e^{at}P_n(t)$, то найдётся решение $y_0(t)=t^k e^{at} R_n(t)$, где $k$ --- кратность корня $a$
\item Если $f(t)=e^{at}(P_n(t)\cos(bt)+Q_n(t)\sin(bt))$, то найдётся решение \\
$y_0(t)=t^k e^{at}(R_n(t)\cos(bt)+S_n(t)\sin(bt))$, где $k$ --- кратность корня $a+bi$
\end{enumerate}
\end{enumerate}

\end{enumerate}
} 
% end of \shpargalka

\begin{document}
\parindent=0 pt % отступ равен 0

Задачи 1-6 оцениваются в 10 баллов, задачи 7 и 8 --- в 20 баллов. При получении 40 баллов оценка за контрольную работу будет гарантированно не меньше 6 баллов. Границы для оценки выше 6 баллов определяются по результатам контрольной работы.

\vspace{20pt}

\begin{enumerate}
\item Найдите первую вариацию функционала $v[y]=\int_{-1}^1 y^{\prime 2} +2y^3 y'+y\sin t \, dt$
\item Найдите допустимую экстремаль функционала
$v[y]=\int_0^2 y^{\prime 2} +3yy'+4y^2+4ye^{2t} \, dt$, при $y(0)=-1$, $y(2)=2(e^4-e^{-4})$

\item Найдите допустимую экстремаль функционала
$v[y]=\int_0^{3\pi/2} y^{\prime 2} -y\cos t -2y\sin 2t \, dt$, при $y(0)=1$

\item Исследуйте на экстремум функционал $v[y]=\int_0^1 e^t \left(y^2+y^{\prime 2}/2 \right) \, dt$, при $y(0)=1$, $y(1)=e$

\item Исследуйте на экстремум функционал $v[y]=\int_0^1 3y^{\prime 4} - 5y^{\prime 3}+25y^{\prime 2}-7y'-6e^t \cos 3t \, dt$, при $y(0)=2$, $y(1)=4$

\item Найдите экстремаль функционала $v[y]=\int_3^5 t^2 y^{\prime 2} \, dt-\frac{2}{3}(y(3))^2+4y(5)$

\item Найдите допустимую экстремаль функционала $v[y]=\int_0^T yt-y^{\prime 2} \, dt$ при $y(T)=-1$

\item Найдите кратчайшее расстояние и соответствующую траекторию между окружностью $t^2+y^2=1$ и прямой $t+y=4$, используя только вариационное исчисление.

\end{enumerate}

\vspace{40pt}

\shpargalka

\newpage
Задачи 1-6 оцениваются в 10 баллов, задачи 7 и 8 --- в 20 баллов. При получении 40 баллов оценка за контрольную работу будет гарантированно не меньше 6 баллов. Границы для оценки выше 6 баллов определяются по результатам контрольной работы.

\vspace{20pt}

\begin{enumerate}
\item Найдите первую вариацию функционала $v[y]=\int_{-1}^1 y^{\prime 2} -4y^3 y'+y\cos t \, dt$
\item Найдите допустимую экстремаль функционала
$v[y]=\int_0^2 y^{\prime 2} -5yy'+9y^2+9ye^{3t} \, dt$, при $y(0)=-1$, $y(2)=3(e^6-e^{-6})$

\item Найдите допустимую экстремаль функционала
$v[y]=\int_0^{3\pi/2} y^{\prime 2} +2y\cos t +4y\sin 2t \, dt$, при $y(0)=1$

\item Исследуйте на экстремум функционал $v[y]=\int_0^1 e^t \left(y^2+y^{\prime 2}/2 \right) \, dt$, при $y(0)=1$, $y(1)=e$

\item Исследуйте на экстремум функционал $v[y]=\int_0^1 3y^{\prime 4} +4y^{\prime 3}+36y^{\prime 2}-7y'-2e^t \sin 3t \, dt$, при $y(0)=3$, $y(1)=2$

\item Найдите экстремаль функционала $v[y]=\int_1^4 t^2 y^{\prime 2} \, dt-\frac{4}{3}(y(1))^2+8y(4)$

\item Найдите допустимую экстремаль функционала $v[y]=\int_0^T yt-y^{\prime 2} \, dt$ при $y(T)=-1$

\item Найдите кратчайшее расстояние и соответствующую траекторию между окружностью $t^2+y^2=1$ и прямой $t+y=4$, используя только вариационное исчисление.

\end{enumerate}

\vspace{40pt}

\shpargalka

\newpage
Задачи 1-6 оцениваются в 10 баллов, задачи 7 и 8 --- в 20 баллов. При получении 40 баллов оценка за контрольную работу будет гарантированно не меньше 6 баллов. Границы для оценки выше 6 баллов определяются по результатам контрольной работы.

\vspace{20pt}

\begin{enumerate}
\item Найдите первую вариацию функционала $v[y]=\int_{-1}^1 y^{\prime 2} +6y^3 y'-2y\sin t \, dt$
\item Найдите допустимую экстремаль функционала
$v[y]=\int_0^2 y^{\prime 2} +4yy'+y^2+ye^{t} \, dt$, при $y(0)=-1$, $y(2)=e^2-e^{-2}$

\item Найдите допустимую экстремаль функционала
$v[y]=\int_0^{3\pi/2} y^{\prime 2} +4y\cos t -8y\sin 2t \, dt$, при $y(0)=1$

\item Исследуйте на экстремум функционал $v[y]=\int_0^1 e^t \left(y^2+y^{\prime 2}/2 \right) \, dt$, при $y(0)=1$, $y(1)=e$

\item Исследуйте на экстремум функционал $v[y]=\int_0^1 -3y^{\prime 4} +4y^{\prime 3}-16y^{\prime 2}-21y'+e^t \sin 3t \, dt$, при $y(0)=-1$, $y(1)=2$

\item Найдите экстремаль функционала $v[y]=\int_2^4 t^2 y^{\prime 2} \, dt-\frac{4}{3}(y(2))^2+6y(4)$

\item Найдите допустимую экстремаль функционала $v[y]=\int_0^T yt-y^{\prime 2} \, dt$ при $y(T)=-1$

\item Найдите кратчайшее расстояние и соответствующую траекторию между окружностью $t^2+y^2=1$ и прямой $t+y=4$, используя только вариационное исчисление.

\end{enumerate}

\vspace{40pt}

\shpargalka

\end{document}