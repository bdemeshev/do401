\documentclass[pdftex,12pt,a4paper]{article}

\input{/home/boris/science/tex_general/title_bor_utf8}

\title{Заметки к динамической оптимизации}

\begin{document}
\parindent=0 pt % отступ равен 0


\[
\text{Оценка}=0.5\text{Накопленная}+0.5\text{Экзамен}
\]

\[
\text{Накопленная}=0.7\text{Контрольная}+0.3\text{Семинары}
\]

\begin{enumerate}
\item Вариационное исчисление
\item Принцип максимуму Понтрягина
\item Принцип Беллмана
\end{enumerate}


\section{Вариационное исчисление}

Функционал. Отображение произвольного множества в множество действительных чисел. 


В нашем случае область определения функционала --- множество функций.


Пример. 
\[
V[p]=\int_0^T \pi(p(t),p'(t))e^{-\rho t}\,dt \to \max_{p(t)}
\]

Пример. $Y=C+J$, $C$ --- потребление, $J$ --- инвестиции

$Y=Q(K,L)$



\[
V[y]=\int_0^T F[t,y(t),y'(t)] \,dt \to \max_{y(t)\in M}
\]

При ограничении, $y(0)=y_0$, $y(T)=y_T$. Константы $y_0$, $y_T$, $T$ фиксированы.


Достаточные условия:
(???)
\begin{enumerate}
\item Интеграл сходится для всех возможных $y(t)$
\item В частности, $y(t)$ непрерывен и дифференциируем
\end{enumerate}


Вариация функции, $\delta_y=y(t)-y_1(t)$


Функции $y(t)$ и $y_1(t)$ близки в смысле близости $n$-го порядка, если 



Свойства интегралов:

\begin{enumerate}
\item $| \int f(t)\,dt| \leq \int |f(t)| \, dt$
\item Интегрирование по параметру
\end{enumerate}


Окрестность, расстояние для функций


Непрерывность, линейность функционала


Вариация функционала


Уравнение Эйлера


Специальные случаи



Лекция 2.

Задача о брахистохроне


Модель Эванса


Модель Тейлора



Функционал от нескольких траекторий


Функционал от производных более высокого порядка. Уравнение Эйлера-Пуассона


Лекция 3.


Задачи с подвижными концами, условие трансверсальности

Модель оптимального спроса на труд


Задача Больца







Тракимус, Основы вариационного исчисления в примерах и задачах




\end{document}